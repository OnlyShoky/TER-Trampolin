\section{Bilan}
\subsection{Bilan individuel}
\subsubsection*{Alexandre de Lingua de Saint Blanquat}
Ce projet m'a beaucoup appris sur le travail de groupe. Au début, je n'étais pas chef de projet, car nous avions choisi d'avoir une organisation sans hiérarchie, mais le projet stagnait et prenait du retard. J'ai pris alors plusieurs initiatives sur le choix des composants et organisé plusieurs réunions avec le groupe et les autres membres ont adhéré et le projet à commencer à décoller. Le problème que nous n’avions pas de fil conducteur, alors je me suis investi pour guider notre équipe. J'espère que le résultat témoigne de mon investissement.

\subsubsection*{Mohamed El-Mourabit}
Mon appréciation globale sur ce projet est très positive. J'ai trouvé très intéressant de pouvoir travailler sur un projet où l'on voit notre objectif final depuis le début. On a eu une complète autonomie tout au long du projet et on a pu développer avec une totale liberté toutes les parties du produit. Notre cliente était très disponible pour répondre à nos questions, ce qui nous a permis de comprendre ses besoins. On a eu un accès total au gymnase pour pouvoir tester nos différents prototypes, et les athlètes étaient tout à fait disposés à nous fournir de l'aide.

\subsubsection*{Hugo Diringer}
Le projet que nous avons eu à faire durant ce semestre a été très enrichissant. Nous avions défini un cahier des charges avec notre cliente puis nous avons eu la liberté d'utiliser les logiciels, langages de programmation et matériel que nous souhaitions pour atteindre nos objectifs. Le fait de démarrer de rien nous a permis d'acquérir de l'expérience et de pouvoir expérimenter pour la première fois la conduite d'un projet du début à la fin en total liberté. De plus, notre cliente était à notre disposition pour nous donner les informations dont nous avions besoin ainsi que pour avoir un trampoline à disposition pour tester notre prototype. Tout cela nous a fortement aidé lors de la réalisation du projet et le fait de pouvoir venir sur place et faire nos tests dans la bonne ambiance était très sympathique. 

\subsubsection*{Hippolyte Catteau}
Mon retour sur le projet est très positifs et surtout très instructif. Tout d'abord nous avons été très bien accueilli par le client et au gymnase. En suivant les horaires donnés nous avons toujours eu accès à un trampoline libre durant notre créneau au gymnase pour pouvoir tester notre produit. Aussi, le projet s'est bien déroulé au sein du groupe grâce a une bonne entente et une confiance mutuelle.
Ensuite d'un point de vue technique, ce qui m'a le plus apporté est le fait d'avoir réaliser un projet de A à Z dans le format "projet d'entreprise". Le client est venu nous voir avec un produit, nous avons établi un cahier des charges ensemble et nous sommes passés par toute les étapes de réalisation d'un projet. Cela nous a permis de mettre en pratique toute nos connaissances scolaire.
Pour finir, nous avons réussir a rendre un produit fonctionnel et respectant le cahier des charges et il n'y a rien de plus satisfaisant d'autant plus que le client avait l'air d'apprécier notre travail.

\subsubsection*{Mahe François}
Ce projet fut très intéressant. C’est la première fois pour moi que j’ai pu être en contact direct avec un cas d’application concret des études que je mène. Il a demandé à l’équipe de nombreuses ressources, et le résultat obtenu et très satisfaisant pour l’ensemble du groupe. Ce projet nous a permis de développer de A à Z un produit complet, répondant, nous l’espérons tous, le mieux possible aux attentes de notre cliente. Le travail en équipe a parfois été délicat à mettre en place, mais malgré des hauts et des bas nous avons tous su trouver les mots pour finalement mener ce projet à bien.