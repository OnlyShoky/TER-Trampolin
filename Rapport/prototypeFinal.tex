\section{Prototype final}
Le prototype final a été testé en condition réelle. Pour faciliter le branchement des capteurs au boitier principal, nous avons utilisé des prises Jack 6.3mm. Un défaut de conception fait que le branchement de ces prises provoque un court circuit très rapide, mais suffisant pour faire redémarre le microcontrôleur. Ce court circuit n'est pas dangereux, car le module d'alimentation est peu puisant, mais il reste à éviter au maximum. C'est pour cela qu'il est recommandé de brancher les capteurs avant l'alimentation au boitier principale. Le test en condition réel nous a permis de tester différents aspects :
\subsection{Installation}
Le but du projet aussi était de rendre le projet accessible à des personnes n'ayant pas de connaissance en électronique. Comme le produit est transportable et peut s'installer sur n'importe quel trampoline nous avons rédiger un manuel d'utilisation avec la partie la plus importante qui est l'installation du produit sur un trampoline. Nous l'avons rédiger sous forme d'étape afin que cette partie sois la plus claire et simple possible. Nous l'avons écrite comme suis:

\begin{itemize}
    \item Brancher les récepteurs lasers aux prises jack du boitier de contrôle.
    \item Mettre la machine sous tension.
    \item Positionner un émetteur laser dans la diagonale du trampoline, positionner les deux autres sur les côtés dans le sens de la largeur du trampoline. Les boitiers sont aimantés et peuvent donc s’attacher au barre métallique du trampoline
    \item Mettre en face les récepteurs lasers. Régler les émetteurs de sorte que le laser arrive bien sur le récepteur laser. A ce moment-là, la lumière derrière le récepteur devrait s’éteindre. 
    \item La machine est prête à l’utilisation
\end{itemize}

\subsection{Mesure du temps}
La mesure du temps est faite pour calculer le temps de vols d'athlète sur un enchainement de dix sauts sur un trampoline. Sachant qu’il faut aussi qu'un membre externe regarde l’athlète pour déclencher le comptage des sauts en appuyant sur le bouton. Le bouton va déclencher ce chronométrage d'une série de 10 mesures en commençant par celle en cours. À chaque fois, le temps de vols de la figure est affiché sur l'écran, ainsi qu'à la fin le score total.
Notre mesure du temps de vols est assez cohérente avec la réalité, avec les tests qu’on a faits sur le microcontrôleur, on arrive à une précision de quelques microsecondes. La seule chose que l’on n’a pas pu tester est la comparaison de nos mesures avec une autre machine de temps de vols, ce qui ne nous permettrait pas de savoir si nos mesures sont pertinentes. 

\subsection{Consultation des données}
La consultation des données est assez réussite. Le score est directement affiché après les sauts sur l'écran et les détails sont sur le site internet. L'ajout de la mémorisation a permis de garder les vingt dernières mesures même si la machine à temps de vol est débranchée. Pour améliorer le produit, il faudrait maintenant ajouter de l'interactivité avec le site internet. Par exemple renommer une mesure, en supprimer une ou en épingler une pour qu'elle reste définitivement. La LED est finalement peu pertinente. Elle peut servir à vérifier que les capteurs fonctionnent bien, mais ne remplace pas la vérification des capteurs individuelle. Il faudrait implémenter un système qui compare les capteurs pour vérifier si l'un d'entre eux est désynchronisé. Néanmoins, pour le temps que nous avions, le résultat reste très satisfaisant.

\subsection{Supports et boîtiers}
Les boîtiers et supports utilisés pour la machine à temps de vol sont fonctionnel mais peuvent encore être améliorer. Nous avons pensé a plusieurs points améliorable à l'avenir tel que le prix et la qualité du produit. Il faut soulever un point important qui l’achat de matériel supplémentaire en plus des impression 3D: des aimants et des vis de serrage rapide. Il serait possible de d'améliorer la qualité du matériel en achetant des produit plus haut de gamme. Il serait aussi possible d'améliorer la qualité des impressions améliorant les dimensions de nos impression pour mieux caler nos aimants (à l'échelle du dixième de mini mètre). Pour finir, nous pourrions réduire la taille du boîtier de contrôle en modifiant la place des composant pour réduire l'espace occupé et rajouter une rondelle auto-bloquante pour stabilisé d'autant plus le laser. Malgré tout ça, nous pouvons être fière d'avoir un produit fonctionnel à l'heure actuel et qui reste en place pendant une durée de plus de 2h sans avoir à re-régler la position des lasers (testé avec des gymnastes).